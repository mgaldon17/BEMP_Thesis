\chapter{Conclusions and Outlook}

The results described in this work show that there is a corrosion layer that has built up over the magnesium surface of the PTW TM33054 ionization chamber. This corrosion layer has developed up to a thickness of 0.37 \unit{\milli\meter}. The composition of this layer is not assessable using the neutron CT scan that has been used to estimate the thickness. For this reason, the layer has been assumed to be made of hydromagnesite, which is a mineral that builds up on magnesium under normal atmospheric conditions. It has been found in the MCNP simulations, that the dose absorbed by the sensitive region of the chamber is about 1.5 times higher when a hydromagnesite corrosion layer is present over the magnesium, providing that the regular MEDAPP source is used. The corrosion layer facilitates the energy absorption under the neutron field of the MEDAPP source. Furthermore, it has been found that the total energy deposited in the gas region always increases linearly with the density of the gas, remaining almost constant when only the neutron source is applied. The presence of water in the corrosion layer plays an important role in the dose. Higher concentration of water leads to a higher dose absorption from both radiation sources. For a water concentration of 0 to 3 $\%$, the doses differ in 10 $\%$ in the case of a neutron source, and 5 $\%$ in case of the regular MEDAPP source.

The corrosion layer is the cause of the increase of the sensibility of the ionization chamber PTW TM33054, which leads to an increased dose. This chamber is assumed to increase its $k_U$ value over the years, and it has been confirmed in the calibration performed in this thesis that the sensibility of the PTW TM33054 chamber has risen from 0.155 to 0.202 since 2018 to 2023. This increase has attributed a shift in the measured dose. 

The chambers TE-13 and TE-14 are assumed to have a constant sensibility. The outcome of the calibration challenges that assumption as the value of $k_T$ has been measured to be 1.031 and 1.038, respectively. There is a build-up effect to account for in the tissue equivalent plastic chambers. However, the effect is negligible for all energies below 2.5 \unit{\mega\electronvolt}.

The change in the neutron sensibility is not visible in the check source measurements.

Future work could focus on an in-depth analysis of the corrosion layer composition through additional chemical studies, offering a more in-depth understanding of the materials involved under varied atmospheric conditions. Long-term sensibility studies are necessary to monitor the evolution of ionization chamber sensibility over extended periods, emphasizing periodic calibration of the chambers to account for the development of the corrosion layer. Additionally, exploring the influence of corrosion layers on the sensibility of the chambers under different energy ranges, especially in scenarios with neutron energies within the range of application for therapy treatments, would contribute to a more comprehensive understanding of the system's response.
