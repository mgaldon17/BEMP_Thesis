\chapter{\abstractname}

This thesis studies the sensibility to neutron and gamma radiation of mainly magnesium ionization chambers used in dosimetry for fast neutron therapy. The sensibility happens to be altered, which results in uncertainties in the dosimetric measurements.

The atmospheric conditions cause the development of a corrosion layer on the PTW magnesium chamber TM33054 that leads to a change in the dose delivered by certain particles. This layer might cause an alteration in the sensibility of the magnesium chamber and result in the unsuitability of the device under a $\beta^-$ check source for daily measurements.

To study the problem, a neutron and a X-ray CT scan were done on the chamber to search for a layer made of another material. A Monte Carlo simulation of the chamber was run after it to investigate how radiation is absorbed per unit mass by its sensitive region in presence of the layer. Additionally, a calibration of the magnesium and tissue equivalent chamber (TE) was performed at Physikalisch-Technische Bundesanstalt (PTB) to calculate their sensibility.

The result of the neutron CT scan confirms the presence of a 0.37 \unit{\milli\meter} thick corrosion layer assumed to consist of hydromagnesite, and the simulations show that the dose absorbed by the magnesium chamber is higher in presence of the layer. A small concentration of water in the layer contributes further to a higher dose deposition. The calibration confirms that the sensibility of the Mg device has increased over the span of 5 years, and challenges the assumption that it remains constant for the TE chambers at 2.5 \unit{\mega\electronvolt}. 

In conclusion, the hydromagnesite layer is the cause of the increased sensibility of the PTW TM33054 magnesium chamber under normal atmospheric conditions. The suitability of the $\beta^-$ check source remains unchanged for check measurements, regardless of the development of a hydromagnesite layer. The increased sensibility of the tissue equivalent chambers is left as a research object for further work.

\cleardoublepage{}
