\chapter{Conclusions and Outlook}

An assessment of the neutron and electron dose absorbed by a specific model of the PTW ionization chamber along with a calibration measurement has been performed throughout this work. Several simulations have been run using the Monte Carlo N-Particle code to reproduce the conditions of the irradiation of the device. The source that has been configured in the dose absorption simulation is the one used at MEDAPP to carry out dosimetry measurements and palliative cancer treatments. 

In conclusion, it has been proved in this work that the PTW 33503 ionization chambers made of magnesium are prone to develop a corrosion layer on the metallic surface. This corrosion layer reaches a thickness of less than half a millimeter when the chamber is placed in an environment with low atmospheric water concentration. The presence of this layer can be detected using a neutron CT scan, and its thickness can be estimated using further image processing techniques. However, the composition of the corrosion layer cannot be acknowledged without an additional chemical analysis of the composite. The simulations of the MEDAPP source on the chamber provide further insight into the response of the device to the presence of photons in the source, and to the concentration of water in the layer. Retrospective research shows that several corrosion materials could be found in the layer, depending mainly on the atmospheric conditions. To this dissertation, hydromagnesite has been accepted as a valid corrosion material.

It has been found in this work, that the dose absorbed by the sensitive region of the chamber is about 1.5 times higher when a hydromagnesite corrosion layer is present over the magnesium, providing that the regular MEDAPP source is used. When the chamber is irradiated by neutrons that keep the MEDAPP configuration, the dose absorption is possible almost only due to the presence of the corrosion layer. Furthermore, it has been found that the total energy deposited in the gas region always increases linearly with the density of the gas, remaining almost constant when only the neutron source is applied. 

The presence of water in the corrosion layer plays an important tole in the dose. Generally, a higher concentration of water leads to a higher dose absorption in both radiation sources. The doses differ in about 10 $\%$ in the case of a neutron source, and 5 $\%$ in case of the regular MEDAPP source for a dose concentration of 0 to 3 $\%$. 

Future work in this domain could delve into an in-depth analysis of the corrosion layer's composition through additional chemical studies, offering a more in-depth understanding of the materials involved under varied atmospheric conditions. Further investigations should expand the scope of dose absorption studies, including diverse ionization chamber materials and irradiation scenarios. The application of advanced and non-destructive imaging techniques may enhance the characterization of corrosion layers, adding depth to the understanding of their structure.

Following the assessment of the atmospheric corrosion of the chamber, the calibration measurements have been performed at PTB to measure the sensibility of the chamber. In addition, a simulation of the so-called build-up effect has been simulated for magnesium and A-150 plastic chambers. In conclusion, the sensibility of the tissue equivalent TE-13 and TE-14 chambers at 2.5 \unit{\mega\electronvolt} has increased minimally over a span of 5 years, and the sensibility of the magnesium chamber has also increased. This change in the sensibility is attributed to the presence of the corrosion layer in the magnesium chamber.

The build-up effect has been proved to be only relevant for neutron energies higher than 2.5 \unit{\mega\electronvolt} and for the A-150 material. For energies higher than that, a significant increase in the dose deposition takes place in the first few millimeters of material. The sensitivity of the PTW chambers addressed in this work has not been influenced by this effect.

Long-term sensitivity studies are necessary to monitor the evolution of ionization chamber sensitivity over extended periods, emphasizing periodic calibration of the chambers to account for the development of the corrosion layer. Additionally, exploring the influence of corrosion layers on the sensibility of the chamber under different energy ranges, especially in scenarios with neutron energies within the range of application for therapy treatments, would contribute to a more comprehensive understanding of the system's behavior. Further attention should be given to the build-up effect, particularly in ionization chambers made of different materials and subjected to varying neutron energy ranges.

To conclude this work, a beta source is employed at MEDAPP to perform daily check calibrations of an ionization chamber due to the spectrum that the strontium-90 sample is able to provide. This source has been simulated carefully taken into consideration the energy bins of the beta and photon radiation spectrum. To simulate this source, the activity of the sample has been calculated on a given date along with their decay probabilities. The same ionization chamber has been placed under this beta source and the dose absorption has been assessed in presence of the corrosion layer. The dose that the gas chamber absorbs has been determined as [...] when it is irradiated by an electron source. This means, that the chamber measures a dose of… with an electron source, and a dose of [...] with the normal MEDAPP source. There is a difference of [...] between them both that can be regarded as relevant to the daily measurements taken with the Sr-90 source. 



