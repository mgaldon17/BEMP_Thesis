\chapter{Introduction}

Over the years there has been a long-standing research of cancer sickness worldwide. Various therapies, both palliative and curative, have been developed to address the disease in different ways. One the principal therapies that has gained the interest of researchers and medical centers is the particle therapy. This therapy is designed and planned by medical physicists in coordination with physicians and consists of irradiating the tissue of interest with a source of particles of different nature, so that the interaction of particle and matter results in a destruction of the malignant cells. Although in radiotherapy treatments several combinations of fundamental particles are susceptible to be used, it is of special interest to look at the case of neutrons only and study their specific interaction with matter. The reason for this decision is the fact that unique therapies that use neutrons for palliative treatments exist in very few places around the globe and succeeded improving the health and life expectancy of the persons who received them. 

Treatments performed with neutrons can be divided in two types: Fast Neutron Therapy (FNT) and Boron Neutron Capure Therapy (BNCT). The main stream of patients have gone for FNT worldwide. For five decades, FNT has been applied to more than 30.000 patients and it could be stablished as a clinical routine therapy that went over the experimental or scientific stage \cite{wagner2015FNTstatus}.
In the world, one of the few facilities that offer FNT is located at FRM II run by the Technical University of Munich (TUM), located in Garching. The term FRM II stands for Forschungsreakor München, which means research reactor in German. Regarding the rest of the facilities, by 2015 only three are still present and running: at the UW Medical Center in Seattle (USA), Polytechnic University in Tomsk (Russia) and FRM II in Garching (Germany) \cite{wagner2015FNTstatus}. The number of facilities decreased over time due to  medical decisions based on the supposed risk of the therapy. 

The medical application facility of FRM II (MEDAPP) supplies a beam of fast neutrons created by fission of $\ce{^{235}{}U}$. The source can be used for medical applications like the irradiation of human cancer, for radiography and CT in non-destructive material characterisation as well as for the irradiation of biological tissue. For such applications, a detailed knowledge of the neutron spectrum is required.

As neutrons cannot directly ionize matter, they need to be converted into charged particles to be detected. The basic physical principles for neutron detection are the neutron’s characteristic properties and several important nuclear reactions. This is a common practice that will be used throughout the project.

The neutron detection will be carried out in this project by a gas-filled detector type, which is the ionization chamber. Their normal operation is based on the collection of the charge created by interaction of radiation in the gas and they are commonly used for medical applications like dosimetry.

The goal of this project is, therefore, to search for corrosion products on the ionization chamber model TM33054 designed by the company PTW, and confirm whether the products affect the reading of the chamber in both clinical applications and daily checks with a $\beta^-$ source. In order to study the response of the ionization chamber, several steps need to be considered. The first is to run a CT scan on the chambers to search for any corrosion product that might affect the measured dose. The scan images were taken at Munich Institute of Biomedical Engineering with the support of Dr. rer. nat. Klaus Achterhold.

The next step is to perform a Monte Carlo simulation of the chamber to measure the dose absorbed by the gas at sensitive region, which is filled by argon gas. The simulation is made using the Monte Carlo N-Particle code (MCNP), which is a software developed by Los Alamos National Laboratory, and is able to generate a wide variety of geometries and irradiation scenarios. The simulation reproduces the mixed neutron and gamma source used at MEDAPP and the geometry of the chamber.

Lastly, a calibration of the chamber is performed at PTB to measure the sensibility of the PTW TM33054 made of magnesium, and of the PTW TM33053 made of tissue equivalent plastic. This last step is done to check whether the sensibility of the device has increased over time.