Magnesium is a reactive material that tends to donate electrons in chemical reactions. In environments with higher moisture levels, magnesium is more likely to undergo oxidation. %since water facilitates the movement of electrons between atoms.

A series of oxidation and reduction reactions occur when magnesium reacts with the water and oxygen present in the atmosphere. These reactions produce numerous hydroxide ions, $OH^{-}$, that give rise to corrosion products. %As with any electrochemical reaction, the rate of product formation is directly influenced by temperature and the relative humidity of the environment.

Martin Jönnson's study is focused on the influence of the microstructure and environment of magnesium alloys. It is demonstrated in his work that the initial corrosion product formed immediately after the oxidation reaction is magnesium hydroxide, $Mg(OH)_2$. 
After several days of exposure to the same temperature and humidity conditions, the final compound found on the alloy surface is \textbf{hydromagnesite} \cite{AtmosphericCorrosionMgAlloys}.

In his work it is concluded that the microstructure of the alloy plays a fundamental role in the corrosion behaviour as well as the presence of $NaCl$ in the air. The conditions that apply to the ionization chambers exclude any amount of sodium chloride that could accelerate the corrosion reactions. The relative humidity of the atmosphere of FRM II is lower than in the location where Martin Jönnsons' research was conducted and the wall of the chamber is made of pure magnesium. Therefore, the corrosion rate of the magnesium, and hence the thickness of the layer, is expected to be lower in the chambers than in the alloys assessed in his work, which is about a few micrometers per year.

While it may not be feasible to identify the corrosion material of the PTW chambers using imaging techniques, the next section demonstrates the presence of a corrosion layer through neutron imaging. Furthermore, an estimation  of the layer thickness is provided, which will be utilized in the coming MCNP simulations.