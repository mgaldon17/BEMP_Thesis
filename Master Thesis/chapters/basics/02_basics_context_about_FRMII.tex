In the past, a total of 715 patients have undergone FNT treatment between 1985 and 2000 at the facilities of FRM, the predecessor of FRM II. The new reactor FRM II became operative in 2005 and the first patient was irradiated at MEDAPP in 2007 \cite{Heinz}. The MEDAPP facility was built to provide similar dosimetric characteristics to the old radiation facility \cite{wagner2015FNTstatus}.

The treatment consists in irradiating slowly growing and/or well differentiated tumors \cite{wagner2008MEDAPP} with fast neutrons at a mean energy of 1.9 \unit{\mega\eV}. FNT is generally accepted as a palliative treatment of shallow tumor lesions \cite{wagner2008MEDAPP} like a carcinoma. 

In order to generate the fast neutrons necessary to carry out the treatment, FRM II is equipped with a converter facility that allows their generation. At the reactor, two converter plates containing 540 \unit{\gram} highly enriched \ce{^{235}{}U} (93\%) \cite{wagner2015FNTstatus} are mounted within the $D_2O$ moderator tank in about 1 \unit{\meter} distance to the reactor core, which powers the converter with thermal neutrons. In the converter facility, fast neutrons are generated by induced fission of the thermal neutrons with the uranium plates. 

The beam that comes out of the converter facility is a mixed radiation field of fast neutrons and gammas. The $\gamma$ component is undesired for the application of FNT. While several filters exist to reduce the $\gamma$-photon contamination, only one can be applied at MEDAPP due to the constraints of the license for medical operations \cite{wagner2008MEDAPP}.

At MEDAPP, neutrons are almost always administrated combined with conventional photon therapy. 
In the following subsection, the neutron physics and neutron interaction with matter will be covered in detail.
