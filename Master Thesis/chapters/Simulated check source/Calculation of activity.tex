In this subsection, the calculations of the activity of the source are discussed. This step is necessary to set the decay energies and probabilities of the source in the simulated setup. 

The production of the \ce{^{90}{}Sr} sample dates from the year 1980 according to \emph{Radioaktive Kontrollvorrichtung 39 Typ 8921 für PTW-Dosimeter} \cite{kontrollvorrichtung}. The exact production date of the radioactive material remains unknown. For the purpose of the calculations, the exact date and initial activity $A_0$, that is taken as a reference to obtain the activity of the sample are \textbf{0.9} $\boldsymbol{mCi}$, this is, 33 \unit{\mega\becquerel}. 

The only decay mode possible for strontium is $\beta^-$ decay. The daughter nucleus \ce{^{90}Y}, has three potential decay modes into \ce{^{90}Zr}, as indicated in the following table sourced from the \textit{International Atomic Energy Agency} \cite{intlAtomicEnergy}:

\begin{table}[!h]
\centering
\begin{tabular}{|l|l|l|l|l|l|l|}
\hline
\rowcolor[HTML]{A9D9C6} 
\multicolumn{1}{|c|}{\cellcolor[HTML]{A9D9C6}\#} & {\color[HTML]{000000} \begin{tabular}[c]{@{}l@{}}\textless{}E_{$\beta$^{-}}\textgreater \\ {[}\unit{\kilo\electronvolt]}\end{tabular}} & \begin{tabular}[c]{@{}l@{}}I_{$\beta$^{-}(abs)} \\  {[}\%{]}\end{tabular} & \begin{tabular}[c]{@{}l@{}}Daughter level \\  {[}keV{]}\end{tabular} & J$\pi$ & \begin{tabular}[c]{@{}l@{}}E_{$\beta$^{-}_ {max}}\\  {[}keV{]}\end{tabular} & Transition type \\ \hline
1                                                & 24.2 5                                                                                                 & 0.0000014                                                   & 2186.27                                                          & 2+ & 92.2                                                        & 1st non-unique  \\ \hline
2                                                & 184.6                                                                                                & 0.0115                                                     & 1760.74                                                           & 0+ & 517.8                                                       & 1st unique      \\ \hline
3                                                & 932.4                                                                                                & 99.9885                                                   & 0                                                                    & 0+ & 2278.5                                                      & 1st unique      \\ \hline
\end{tabular}
\caption{Decay paths of \ce{^{90}_{39}Y} according to the International Atomic Energy Agency.}
\label{table: decay paths of yttrium-90} %according to the International Atomic Energy Agency
\end{table}

The decay listed as #\emph{$1$} has such a very low absolute probability that the yttrium decay path into the most excited state of zirconium has been neglected. It is assumed from this point, that there will be a total of \textbf{three $\boldsymbol{\beta}^-$} and only \textbf{one $\boldsymbol{\gamma}$} emission in the entire decay chain. Those are the decays that have been included in the MCNP model of the \ce{^{90}{}Sr} source. 

Regarding the $\beta^-$ decays, these are included in the model only after having the spectra of the strontium and yttrium sampled accordingly to account for the continuous nature of the $\beta^-$ decay. The $\gamma$ emission is considered separately in MCNP and only one energy is regarded in our problem as it is a discrete event. Further explanations about the separation and configuration of the energies are given later in this chapter.

From nuclear physics it is known that a parent-daughter nuclear decay chain is ruled by a system of two differential equations describing the particle production rate of the parent and daughter nucleus whose solution \cite{NUK3Notes} for the daughter nucleus is:

\begin{align}
\label{eq:Number of daughter nucleus as a function of time}
    {N_{D}(t) = \frac{\lambda_{P} N_{P} (0)}{\lambda_{D}-\lambda_{P}} (e^{-\lambda_{P} t} -e^{-\lambda_{D} t} )+ N_{D}(0)} \: e^{-\lambda_{D} t}
\end{align}

From Equation \ref{eq:Number of daughter nucleus as a function of time}, and taking into account that the half life of the parent is much longer than that of the daughter, it can be concluded that the decay chain reaches a situation known as \textbf{secular} equilibrium. This means that both activities reach a saturation activity after a very long time. This means that the activity of the daughter can be calculated as:

\begin{align}
\label{eq:Activity of daughter nucleus as a function of time}
    {A_{D}(t) = -\frac{dN_{D}}{dt} =  \lambda_{D}N_{D}(t)}
\end{align}

In Equation \ref{eq:Activity of daughter nucleus as a function of time}, the term $\lambda_{P}N_{P}(t)$ is not stated because it is neglected as the quantity is much smaller than the term $\lambda_{D}N_{D}(t)$ as $\lambda_{D} > \lambda_{P}$. Here, the parameters of Equations \ref{eq:Activity of daughter nucleus as a function of time} and \ref{eq:Number of daughter nucleus as a function of time} are the same, where $\lambda_{P}$ and $\lambda_{D}$ stand for the radioactive decay constants calculated from the half lives of the parent (strontium) and daughter (yttrium), respectively. $N_{P} (0)$ stands for the initial number of nucleus of the strontium. For the strontium sample, i.e. the parent nucleus, the activity is simply calculated according to the radioactive decay law:

\begin{align}
\label{eq:Activity of parent nucleus as a function of time}
    {A_{P}(t) = \lambda_{P} N_{P} (0) e^{-\lambda_{P} t}} 
\end{align}

Taking as the reference date for the calculation of the activities from \ref{eq:Activity of parent nucleus as a function of time} and \ref{eq:Activity of daughter nucleus as a function of time} the $5^{th}$ of July of 2023, it results in 12.02 \unit{\mega\becquerel} and 12.018 \unit{\mega\becquerel}, respectively. As it is expected from the secular equilibrium, both samples have approximately the same activity.